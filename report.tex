\documentclass[12pt,oneside,a4paper]{article}
\usepackage[utf8]{inputenc}
\usepackage[hidelinks]{hyperref}
\usepackage[english]{babel}
\usepackage{indentfirst}
\usepackage{tocloft}
\renewcommand{\cftsecleader}{\cftdotfill{\cftdotsep}}
\renewcommand{\baselinestretch}{1.5}
\def\labelitemi{--}

\usepackage{geometry}
 \geometry{
    a4paper,
    left=20mm,
    right=10mm,
    top=20mm,
    bottom=20mm,
 }

\let\oldenumerate\itemize
\renewcommand{\itemize}{
  \oldenumerate
  \setlength{\itemsep}{0pt}
  \setlength{\parskip}{0pt}
  \setlength{\parsep}{0pt}
}

\title{Openlab - Automated testing and evaluation platform of the source code from assignments}
\author{Mihai Iachimovschi}
\date{}

\begin{document}
\maketitle
\clearpage

\renewcommand*\contentsname{Table of contents}
\tableofcontents
\clearpage

\listoffigures
\clearpage
 
\listoftables
\clearpage

\section{Project Analysis and System Requirements}
\subsection{Project analysis}
Openlab is a platform which aims to provide automated testing and evaluation of students based on their source code from the assignments. The goal of the system is to offer the possibility to organize the assignments for any subject in order to make them easily accessible for both students and teachers. It offers a transparent layer for setting strict deadlines, penalization for late submissions and even rewards for early submissions. Also, speaking of transparency, for any assignment there is a open grading policy which includes the requirements list and the points a student can achieve. Thus, it is intended to improve the efficiency coefficient of students and professors.

The system has a web interface which makes it extremely flexible and cross-platform. The UI is simplified as much as possible so it can be used by anyone without additional training. From the student's perspective, the system will run on his code a prepared test case that was written by the professor and will publish it's results. All the external code should be executed in a completely isolated environment in order to protect the server from malicious code that can be submitted.

Conceptually speaking, the core of Openlab can be described as a platform which can accommodate fully isolated containers that compile and execute all the provided source code, saving the results to a database.

\subsubsection{Problem description}
Students from the IT faculty are dealing with a lot of laboratory works, homeworks and individual assignments for which they should implement different algorithms by writing small programs. For each subject, the requirements are presented in varying forms which makes a inconsistent work-flow for both students and teachers.

Scheduled laboratory sessions become focused on verification and evaluation of students' past assignments. This activity is very time consuming for the professor and it turns out that almost all the time reserved for the lab is consumed by this important but time-wasting procedure.

Automated testing and evaluation of the source code can obviously improve the learning process by focusing on more important and useful tasks. Professors can spend more time on explaining different approaches, technologies or algorithms that can be used for the assignments instead of verifying past assignments in the time reserved for the lab class.

People tend to procrastinate, that's why a lot of students submit their assignments late. Not having a obvious overview of the deadlines can be misleading for students. Being confused may leave space for excuses. That's why enforcing transparent deadlines,notifications and bonuses for early submissions may improve student's punctuality which is useful not only in university.

\subsubsection{Overview of similar products}
Most of the similar projects position themselves as LMS (Learning Management System). The main goal of learning management systems is the delivery of electronic educational materials to the students. It offers also to the professors a way to administer and document courses and also track students' results by evaluating assignments and tests.

\textbf{Moodle} is a flexible, free software, open-source learning platform. It is a LMS that offers basic assignment submission features, forum for discussions, glossary of definitions, a taxonomy for courses and lectures and on-line quiz module. Moodle is written in PHP and can be deployed on any server capable of running PHP and it requires a web server such as Apache or NGiNX. It is compatible with MySQL, PostgreSQL and other relational database management systems.

\textbf{DigitalChalk} is a proprietary Learning Management System which allow users to create and deliver their courses materials on-line. The solution is provided to the customers as a SaaS (Software as a Service). The price starts at 4.95\$ per month per user with an initial setup fee starting at 399\$.

\textbf{Blackboard Learning System} is a complex proprietary Learning Management System. It is highly customizable and it can be either deployed on local server or used as SaaS. The company's pricing policy is not publicly available, however it is rated as expensive.

All considered applications are great tools, highly functional and useful, but none of them has the possibility of automatic testing and evaluation of source code for the assignments which is very important for IT faculty. 

Some of the similar products offer too much overhead in terms of unneeded functional while others are too expensive for a non-profit organization. Openlab aims to provide a minimalistic and simplistic approach to the problem solving. It is also Free and Open Source Software.

\subsection{System requirements}
\subsubsection{Product functions}
This section provides a brief overview of the functions that system will perform. Below are presented the key functions of the product which must be implemented in the final version. The whole system should have the following features:
\begin{itemize}
  \item Keep track of laboratory assignments for users (students);
  \item System should accept assignment submissions and evaluate them;
  \item System should provide an interface for professors for adding new assignments and test cases;
  \item Execute safely the submitted code and the test and save the results;
  \item Grade automatically the submitted and executed assignments;
  \item Inform students about last changes via a private feed;
  \item Display upcoming assignments for logged in student;
  \item Display upcoming laboratory sessions;
  \item Display enrolled courses that registered in the system;
  \item Keep track of grades from the automatically graded assignments;
\end{itemize}

\subsubsection{Constraints and dependencies}
This sections presents basic system requirements for the whole ecosystem. A client that uses the applications have to have a modern web browser installed on his system. On the back-end side, constraints and dependencies are the following:
\begin{itemize}
  \item Server should run any Linux distribution with kernel newer than 3.13; 
  \item Python 2.7 or newer should be installed;
  \item Python PIP should be installed. It will provide packages as: fig, virtualenv, etc.; 
\end{itemize}

\section{Software Modeling}

\section{System Implementation}

\end{document}